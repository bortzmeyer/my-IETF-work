\documentclass[ignorenonframetext]{beamer}
\usetheme{Madrid}
\usepackage[latin1]{inputenc}
\usepackage{bortzmeyer-utils}

\title{QNAME minimisation, the next (last?) steps}
\author{St�phane Bortzmeyer - AFNIC}
\date{IETF 92 - Dallas}

\begin{document}

\begin{frame}
  \titlepage
 Three small questions
\end{frame}

\begin{frame}
\frametitle{Negative consequences}
\begin{itemize}
\item Some people
  \url{http://blogs.verisigninc.com/blog/entry/minimum_disclosure_what_information_does}
  says that qname minimisation minimises the amount of data sent (yes,
  that's the goal). 
\item Privacy, like every security feature, is always a compromise. 
\item \emph{The negative consequences are today presented at the
beginning of section 3 of the draft -02. Is this mention sufficient or should we add something?}
\end{itemize}
\end{frame}

\begin{frame}
\frametitle{Hiding the qtype? Issue \#9, from Shumon Huque}
\begin{itemize}
\item Current text in -02 hides the qtype behing a generic QTYPE=NS
  (or A as a fallback for some broken boxes).
\item Auth. name servers can no longer see (DSC) and publish data about qtypes
  (AAAA/A, prevalence of SRV\ldots) 
\item Should we preserve the original qtype?
\end{itemize}
\end{frame}

\begin{frame}
\frametitle{Keep optimisations?}
\begin{itemize}
\item Draft -02 mentions two algorithms, an aggressive one (never send
  full qnames) and a lazy one (underspecified, leaks more data). 
\item Should we keep both? Or just the aggressive one?
\item Same thing for other possible
optimisations like treating the root and the TLDs in a special way?
\item Is it even necessary since a resolver can optimise at will
  (unilateral decision)?
\end{itemize}
\end{frame}

\end{document}
